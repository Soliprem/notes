\documentclass{article}
\usepackage{titlesec}
\usepackage{titling}
\usepackage{hyperref}
\author{Francesco Prem Solidoro}
\title{Relazione PCTO}

\begin{document}
  \maketitle
  \tableofcontents
  \section{ISIG}
  L'I.S.I.G (Istituto di Sociologia Interazionale di Gorizia) è un'instituzione culturale fondata nel '68. Si occupa di ricerca, consulenzia, studio e formazione per lo sviluppo locale, la cooperazione intenazionale e la convivenza pacifica. Con l'ISIG abbiamo svolto 5 giorni di attività, per un totale di 12 ore:
  \begin{enumerate}
    \item \underline{1° giorno}: Individuazione dei criteri di scelta delle "capitali europee della cultura", introduzione ad ISIG.
    \item \underline{2° giorno}: Ideazione e inizio progettazione di un questionario da somministrare agli studenti, riguardante le loro abitudini culturali.
    \item \underline{3° giorno}: Somministrazione del questionario ad un campione di studenti.
    \item \underline{4° giorno}: Attività domestica di inserimento dei dati ottenuti.
    \item \underline{5° giorno}: Discussione ed interpretazione dei dati. Proposte riguardo aumentare l'afflusso a diversi eventi culturali
  \end{enumerate}
  \section{CRS e Protezione Civile}
  \subsection{OGS-CRS}
  CRS (Centro Ricerca Sismologica), branca dell'OGS (Ocenografia Geofisica Sperimentale), si occupa di ricerca riguardo argomenti quali: sismicità naturale, pericolosità sismica, sismicità indotta, monitoraggio della sismicità in vari posti (quali l'Antartide), la determinazione della sismicità locale e lo studio delle deformazioni crostali da GPS. Sotto il tutoraggio di Carla Barnaba e Antonella Peresan, abbiamo affronatato i temi come lo studio della sismicità, pericolosità sismica e determiazione della risposta sismica locale, con una particolare attenzione prestata al Terremoto del Friuli del '76. Le lezioni si sono tenute nel corso della giornata del 13 Gennaio.
  \subsection{Protezione Civile} % (fold)
  La protezione civile è un ente publico, la cui principale funzione è la gestione delle emergenze locali e nazionali. Per fare ciò pone una particolare enfasi sull'organizzazione delle varie branche che si occupano delle situazioni di emergenza: ad esempio, è la protezione civile a gestire il 112.

  Con la protezione civile abbiamo fatto un percorso divulgatorio su quelle che sono le funzioni che ricopre, a partire dal rischio sismico. Segue un elenco delle attività delle singole giornate. 
  \begin{enumerate}
    \item \underline{1° giorno}: Recatici alla sede della protezione civile (Palmanova), abbiamo assistito ad una presentazione riguardante la struttura, l'apparato della protezione civile ed i ruoli dei suoi reparti. Ci è poi stato parlato dellla risposta al sisma del '76
    \item \underline{2° giorno}: Sempre a Palmanova, abbiamo seguito una lazione sull'importanza della comunicazione con cittadini e volontari. Abbiamo poi visitato gli uffici del 112.
    \item \underline{3° giorno}: Esercitazione divisi in gruppi. Simulazione della risposta ad un sisma (ogni gruppo rappresenta una delle funzioni essenziali alla risposta)
    \item \underline{4° giorno}: Nella sede centrale del Buonarroti (via Matteotti, 8, Monfalcone), abbiamo tenuto una presentazione alla classe 4ASA riguardante la nostra esperienza.
   \end{enumerate}
   \section{Corso sulle geometrie non euclidee}
   In Giugno 2019 ho seguito un corso tenuto dall'università di Udine sulla geometrie non euclidee. Le competenze acquisite includono:
   \begin{enumerate}
     \item \underline{digitali}: Geogebra e MS Teams
     \item \underline{nozionistiche}: Nozioni di base su geometrie diverse da quella euclidea: sono state utili in futuro da un punto di vista inetrdisciplinare per comprendere il relativismo morale, e introdurmi a paradigmi diversi da quelli che giò conoscevo.
   \end{enumerate}
   L'esperienza è stata tutto sommato produttiva e piacevole

   \section{Freedom of Press}
   Attività di sensibilizzazione sulla libertà di stampa nel mondo. L'esperienza è stata formativa in quanto ha sviluppato molto il mio interesse per gli avvenimenti di attualità: ci è stato infatti concesso di partecipare a conferenze molto interessanti. Inoltre mi ha introdotto alle difficoltà ed i conflitti di interesse che caratterizzano la vita del giornalista: la spinta verso il non dire la verità che viene dall'esterno contro un desiderio di dirla (incontro fatto con un giornalista d'inchiesta). Tutto sommato è stata un'attività che ha promosso pensiero critico e riflessione sul presente, il che è certamente positivo.

   \section{Conclusione e commento personale}
  Non ho trovato particolare utilità nelle attività di PCTO svolte in questo modo: aldifuori dell’aspetto nozionistico (ad esempio imparare ad usare geogebra, o come si calcola il rischio sismico), sono pochi I benefici esclusivi del PCTO. Ho menzionato lo sviluppo di un pensiero critico, ma questo viene costantemente proposto anche in classe, nelle lezioni normali. 

Concentrandosi quindi sul nozionismo: la scuola segue programmi che meglio permettono di apprendere le cose in modo congruo e coeso. La maggior parte delle attività svolte parlavano di argomenti in maniera divulgativa che non permette di formare opinioni basate sui fatti, cosa che reputo sempre più importante dato il corrente stato dell’industria culturale, ma su come I fatti vengono descritti e raccontati. Dovrebbe essere la funzione del liceo fornire sistemi attraverso I quali interpretare il mondo in modo critico e personale, e non penso che il PCTO promuova questa visione.

Un possibile miglioramento sarebbe cercare di inserire gli studenti in attività più strettamente collegate al mondo del lavoro a livello individuale, e confido che il PCTO, essendo un progetto relativamente giovane, riuscirà a trovare il suo posto nella formazione dei futuri cittadini.

\end{document}
