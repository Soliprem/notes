\documentclass{article}
\title{Heidegger}
\author{Francesco Prem Solidoro}
\usepackage{biblatex}
\addbibresource{soli_list.bib}
\usepackage{hyperref}

\begin{document}
\maketitle
\tableofcontents
\section{Vita} % (fold)
\label{sec:VIta}

% section section name (end)
\section{Filosofia} % (fold)
\label{sec:Filosofia}
\subsection{Dasein} % (fold)
\label{sub:Dasein}
\textcite{sein_und_zeit_heidegger} in \textit{Sein und Zeit} rifiuta la visione Cartesiana dell'essere umano come un osservatore soggettivo di oggetti ~\cite{Horrigan-Kelly_Heidegger}. Il libro infece sostiene che il soggetto e e l'oggetto siano inseparabili. Presentando l'essere come inseparabile, Heidegger introduce il termine \textit{Dasein} (letteralmente traducibile con "essere lì"), con cui vuole rappresentare un "essere vivente" attraverso il suo ``essere lì" e ``essere-nel-mondo". L'essere-nel-mondo è una caratteristica fondamentale del Dasein.

La rappresentazione dei \textit{Dasein} in \textit{Sein und Zeit} viene articolata attraverso un'osservazione dell'esperienza dell'angoscia, "il niente" e la mortalità, e poi attraverso l'analisi della struttura della ``proccupazione". Da lì, Heidegger propone il problema dell' "autenticità", cioè il potenziale del Dasein \textit{Dasein} mortale di esistere abbastnza appieno, da poter comprendere l'essere e le sue possibilità. \textit{Dasein} non è ``l'uomo", ma niente se non l'uomo è \textit{Dasein}. Inoltre, Heidegger scrive che il \textit{Dasein} è ``l'essere che darà accesso alla questione del significato dell' essere" ~\cite{Peter_caws_Heidegger_Sartre}
% subsection subsection name (end)

\subsection{Essere} % (fold)
\label{sub:Essere}
L'esperienza ordinaria e mondana di Dasein del' ``essere-nel-mondo" da ``accesso al significato" o al ``senso dell'essere" 

% subsection subsection name (end)


% section section name (end)
\printbibliography
\end{document}
