\documentclass{article}
\author{Francesco Prem Solidoro, per conto di Samuele Marro}
\title{Bargaining made Instant}
\begin{document}
\maketitle
Obiettivo: avere una forma di bargaining tra Alice e Bob dove la divisione del valore aggiunto è equa. In particolare si ottiene che l'utilità di ogni parte è massima quando dice la verità sul prezzo

Per esempio, la variante del quadratic bargaining assume utilità lineare: Alice ha utilità ax, Bob ha utilità bx, dove x è la quantità di beni

Alice compra s beni e trasferisce 
$$1/2 * s^2 dollari a Bob$$

Bob compra t beni e trasferisce 
$$1/2 * t^2$$
dollari ad Alice

Alice ha utilità 
$$a(s - t) - 1/2 s^2 + 1/2 t^2$$
L'unico parametro che Alice può controllare è s, quindi derivando l'utilità per s abbiamo
$$d/ds = a - s$$
La derivata è 0 per s = a, e dato che è una parabola al contrario è un massimo globale

Per Bob è simmetrico, la sua utilità è massima quando t = b

L'utilità di Alice è quindi 
$$a(a - b) - 1/2 a^2 + 1/2 b^2 =$$ 
$$a^2 - ab - 1/2 a^2 + 1/2 b^2 =$$
$$1/2 a^2 + 1/2 b^2 - ab$$ 
$$= 1/2 (a - b)^2$$

Bob simmetrico, esce 

$$1/2 (b - a)^2$$

Entrambe le parti ottengono la stessa utilità, e in nessun momento hanno dovuto essere onesti

Due note:
- Sto lavorando per estenderlo supponendo che l'utilità w.r.t. la quantità di beni non è lineare. Per il momento ho fatto utilità logaritmica e alcune monomiali
- La scala dell'unità di misura di a e b non è fissa: variandola cambia la quantità di beni trasferiti (se il prezzo è $/kg o $/t cambia). In altre parole, l'algoritmo non è scale-invariant: è possibile scegliere la scala tale che il trasferimento di beni sia pari ai beni in magazzino di Alice o di Bob (in base a chi trasferisce al netto)
\end{document}
