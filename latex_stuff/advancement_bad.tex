\documentclass{article}
\title{Why advancement bad}
\author{Francesco Prem Solidoro}
\usepackage[backend=biber]{biblatex}
\addbibresource{soli_list.bib}
\usepackage{}

\begin{document}
\maketitle
\section{Pirandello} % (fold)
\label{sec:Pirandello}
\subsection{Il fu Mattia Pascal} % (fold)
\label{sub:Il fu Mattia Pascal}
ne \textit{Il Fu Mattia Pascal}, \textcite{il_fu_mattia_pascal} presenta nel capitolo in cui Adriano Meis si reca a Milano un esemprio si come lo sviluppo non sempre corrisponda ad un miglioramento della condizione umana.\\
Si presenta un personaggio indigente, che però trae grande piacere dal fare giri in tram. Non ne avrebbe un naturale bisogno, nè una naturale inclinazione.
% subsection :Il fu Mattia Pascal (end)
\subsection{I quaderni di Serafino Gubbio} % (fold)
\label{sub:I quaderni di Serafino Gubbio}
% subsubsection subsubsection name (end)
\subsection{Uno, Nessuno, Centomila} % (fold)
\label{sub:Uno Nessuno Centomila}
% subsection subsection name (end)


% section section name (end)
\section{Leopardi} % (fold)
\label{sec:Leopardi}
\subsection{Discorso di un Italiano sulla Poesia Moderna} % (fold)
% subsection subsection name (end)
\label{sub:Discorso di un Italiano sulla Poesia Moderna}
\subsection{Lo Zibaldone} % (fold)
\label{sub:Lo Zibaldone}
% subsection subsection name (end)


% section section name (end)
\section{Montale} % (fold)
\label{sec:Montale}
\subsection{Discordo Premio Nobel} % (fold)
\label{sub:Discordo Premio Nobel}
% subsection subsection name (end)


% section section name (end)
\section{Calvino} % (fold)
\label{sec:Calvino}
\subsection{Il Mare dell'oggettività} % (fold)
\label{sub:Il Mare dell'oggettività}
% subsection subsection name (end)


% section section name (end)
\printbibliography
\end{document}
