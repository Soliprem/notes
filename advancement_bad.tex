\documentclass{article}
\title{Contro lo Sviluppo}
\author{Francesco Prem Solidoro}
\usepackage{color}
\usepackage{hyperref}
\hypersetup{
    colorlinks,
    citecolor=black,
    filecolor=black,
    linkcolor=black,
    urlcolor=black
}
\usepackage[backend=biber]{biblatex}
\addbibresource{soli_list.bib}
\usepackage{}

\begin{document}
\maketitle
\tableofcontents
\section{Pasolini} % (fold)
\label{sec:Pasolini}
\subsection{La differenza fra Progresso e Sviluppo} % (fold)
\label{sub:La differenza fra Progresso e Sviluppo}

% subsection subsection name (end)

% section section name (end)
\section{Pirandello} % (fold)


\label{sec:Pirandello}
\subsection{Il fu Mattia Pascal} % (fold)
\label{sub:Il fu Mattia Pascal}
ne \textit{Il Fu Mattia Pascal}, \textcite{il_fu_mattia_pascal} presenta nel capitolo in cui Adriano Meis si reca a Milano un
esemprio si come lo sviluppo non sempre corrisponda ad un miglioramento della condizione umana.\\
Si presenta un personaggio indigente, che però trae grande piacere dal fare giri in tram.
Non ne avrebbe un naturale bisogno, nè una naturale inclinazione, eppure è eccitato nel vedere il potenziale che 
lo sviluppo tecnologico porta.\\
Questo è un classico esempio dell'umorismo Pirandelliano come descritto nel trattato
\textit{L'Umorismo:}\cite{umorismo_pirandello} un elemento grottesco che inizialmente porta all'ilarità, ma in un secondo momento
ad una riflessione sulla propria condizione: in questo caso l'autore spinge a considerare come la necessità dell'essere umano di
stare al passo dello
sviluppo tecnologico, spendendo (in questo caso) denaro che non si può permettersi di spendere per fare un giro in tram.
E' più importante l'essere omologati alla società di essere autentici o felici.

% subsection :Il fu Mattia Pascal (end)
\subsection{I quaderni di Serafino Gubbio} % (fold)
\label{sub:I quaderni di Serafino Gubbio}
Nel romanzo \textit{I Quaderni di Serafino Gubbio Operatore}\cite{quaderni_serafino}, Pirandello esprime la voce di un impiegato di fabbrica. In queste veci, muove una critica a quello che è il sistema di sfruttamento sistematico che opprimeva la classe operaia del tempo: Serafino, in un passo climatico dell'opera, viene ridotto ad una mano che gira una manovella. Un Uomo al servizio della macchina. In questa posizione servile, Serafino rappresenta la classe operaia, schiava della macchina che scandisce un ritmo sempre più serrato, e rende il lavoratore sempre più facile da rimpiazzare, garantendo un migliore plus valore per il proprietario, alle spese del lavoratore. Inoltre, l'assimilazione dell'essere umano alla "mano" si rifà al concetto di maschera: la funzione sociale dell'individuo (l'operatore) assorbe l'intera persona.

% subsubsection subsubsection name (end)
\subsection{Uno, Nessuno, Centomila} % (fold)
\label{sub:Uno Nessuno Centomila}
Da definire: confronto da fare fra la condizione iniziale, in equilibrio precario che viene rotti dalla realizzazione del
naso storto, da confrontare con la peripezia e il finale idilliaco al di fuori della società. Trovare un passo "forte"
% subsection subsection name (end)


% section section name (end)
\section{Leopardi} % (fold)
\label{sec:Leopardi}
\subsection{Discorso di un Italiano sulla Poesia Moderna} % (fold)
Leopardi ripropone la visione Vicana del tempo che associa a diversi periodi storici diverse età: l'infanzia corrisponde
all'antichità, l'adolescenza al romanticismo, l'età adulta all'illuminismo; l'andamento delle ere è ciclico. 
% subsection subsection name (end)
\label{sub:Discorso di un Italiano sulla Poesia Moderna}
\subsection{Lo Zibaldone} % (fold)
\label{sub:Lo Zibaldone}
% subsection subsection name (end)


% section section name (end)
\section{Montale} % (fold)
\label{sec:Montale}
\subsection{Discordo Premio Nobel} % (fold)
\label{sub:Discordo Premio Nobel}
% subsection subsection name (end)


% section section name (end)
\section{Calvino} % (fold)
\label{sec:Calvino}
\subsection{Il Mare dell'oggettività} % (fold)
\label{sub:Il Mare dell'oggettività}
% subsection subsection name (end)
\section{Verga} % (fold)
\label{sec:Verga}
\subsection{La fiumana del progresso} % (fold)
\label{sub:La fiumana del progresso}

% subsection subsection name (end)

% section section name (end)


% section section name (end)
\section{Marx} % (fold)
\label{sec:Marx}
\subsection{Das Kapital} % (fold)
\label{sub:Das Kapital}
% subsection subsection name (end)
\subsection{Die deutsche Ideologie} % (fold)
\label{sub:Die deutsche Ideologie}
% subsection subsection name (end)

% section section name (end)

\printbibliography
\end{document}
