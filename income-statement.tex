\documentclass{article}
\title{Enel Income Statement}
\author{Francesco Prem Solidoro}
\begin{document}
\maketitle
\tableofcontents
All values used in this analysis, in accordance with the sources used, will be expressed in thousands of dollars, and represent the state of the company in the fiscal year of 2022
\section{Balance Sheet}
The following analysis of \textit{Enel's} Balance sheet and Income statement aims to extract information about the trajectory and financial situation of the company.
\subsection{Assets}
By analysing the current assets, we are going to be able to determine that both current and non current assets grew. It will therefore be no surprise that total assets, which amount to 219618, reported a growth of 6.13\% when compared to 2021.
\subsubsection{Non-Current Assets}
Analysing \textit{ non current assets } we can point out that \textit{ intangible assets } amounts to 31262, which is the worth of non-monetary items used and designed to produce future growth.\\
\textit{ Tangible assets }, which this year amounted to 8635, are another component of Non-Current Assets, which indicate goods that are not meant to be sold within the next business cycle; thei aim is economic growth. The previous year, they amounted to 4560, for an overall net positive growth of 4075 (89.36\%).\\
The value of physical capital owned this year (\textit{net Property, Plants and Equipment}) amounts to 88521\\
Lastly, the value of \textit{Total Investment and Advances} is equal to 14704. \\
These values and other invoices contribute to a total value for \textit{Non Current Assetes} of 153555, an increase of 11747 compared to last year.
\subsubsection{Current Assets}
Current Assets are defined by being quickly convertable into liquidity, and they include:
\begin{itemize}
  \item \textit{ Cash and Equivalents }, which are the most liquid of all current assets and carry close to no risk of devaluation. This year they amounted to 11041, a growth of 23.42\% compared to the last fiscal period.
  \item \textit{ Recievables }. The value of this assets decreased by 648 compared to 2021, to reach the value of 17272.
  \item \textit{Inventories}, which are goods stored for sale within the company's operating cycle, amount to 4853, underlying a growth of 1418 compared to 2021.
  \item \textit{Other Current Assets} make up 32897 of hte total \textit{Current Asset's} value, amounting to 66063.

\end{itemize}
\subsection{Liabilities}
\subsubsection{Non-Current Liabilities}
The analysis of liabilities is threfore reported starting off with non current liabilities, which include:
\begin{itemize}
  \item \textit{Deferred Tax Liabilities}: taxes that are owed by the company, but are not required to be paid in the short term. They amount to 1383
  \item \textit{Provision}: funds set aside to cover anticipated future expenses. Equal to 8257
  \item \textit{Long Term Debt} amounts to 68191.
  \item \textit{Deferred Income}, which equlals to 5747
\end{itemize}
These and other entries make up for the total value of non-curret liabilities: 105238. an increase of 10831 (11.47\%) compared to 2021.
\subsubsection{Current Liabilities}
Moving on with the last part of the liability analysis, we'll consider current liabilities.\\
They are composed as follows:
\begin{itemize}
  \item \textit{Short Term Debt, and current portion long term Debt}: these are equal to 21227
  \item \textit{Accounts Payable}: an invoice in which a 4.02\% growth was recorder, bringing the total amount to 17641
  \item \textit{Income Tax Payable}: money owed by the company to the state for the payment of income taxes, are equal to 1623
  \item \textit{Other Current Liabilities}: this invoice amounts to 31807 
\end{itemize}
The total amount of current liabilities amounts to 72298, highlighting a decrease in this invoice of 3460 in the last fiscal year.\\
Nontheless, due to the increase in non current liabilities, total liabilities increased by 7371 compared to the previous year, as their value is equal to 177536.
\subsection{Equity}
The final part of the balance sheet analysis will focus on equity: this year retained earnings make up for 15797 of the total equity, which is equal to 42082. The latter gerw by 14.43\% in comparison to 2021, when it amounted to 36775.\\
In conclusion, we can state that the total amount of liability and equity is equal to 219618, which coincides to the value of total assets. We can also note how a growth in both equity and liabilities resulted in a 6.13\% increase in total assets.
\section{Income Statement} 
\end{document}
