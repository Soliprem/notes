\documentclass{article}
\title{Enel Income Statement}
\author{Francesco Prem Solidoro}
\begin{document}
\maketitle
\tableofcontents
All values used in this analysis, in accordance with the sources used, will be expressed in thousands of dollars, and represent the state of the company in the fiscal year of 2022
\section{Balance Sheet}
The following analysis of \textit{Enel's} Balance sheet and Income statement aims to extract information about the trajectory and financial situation of the company.
\subsection{Assets}
By analysing the current assets, we are going to be able to determine that both current and non current assets grew. It will therefore be no surprise that total assets, which amount to 219618, reported a growth of 6.13\% when compared to 2021.
\subsubsection{Non-Current Assets}
Analysing \textit{ non current assets } we can point out that \textit{ intangible assets } amounts to 31262, which is the worth of non-monetary items used and designed to produce future growth.\\
\textit{ Tangible assets }, which this year amounted to 8635, are another component of Non-Current Assets, which indicate goods that are not meant to be sold within the next business cycle; thei aim is economic growth. The previous year, they amounted to 4560, for an overall net positive growth of 4075 (89.36\%).\\
The value of physical capital owned this year (\textit{net Property, Plants and Equipment}) amounts to 88521\\
Lastly, the value of \textit{Total Investment and Advances} is equal to 14704. \\
These values and other invoices contribute to a total value for \textit{Non Current Assetes} of 153555, an increase of 11747 compared to last year.
\subsubsection{Current Assets}
Current Assets are defined by being quickly convertable into liquidity, and they include:
\begin{itemize}
  \item \textit{ Cash and Equivalents }, which are the most liquid of all current assets and carry close to no risk of devaluation. This year they amounted to 11041, a growth of 23.42\% compared to the last fiscal period.
  \item \textit{ Recievables }. The value of this assets decreased by 648 compared to 2021, to reach the value of 17272.
  \item \textit{Inventories}, which are goods stored for sale within the company's operating cycle, amount to 4853, underlying a growth of 1418 compared to 2021.
  \item \textit{Other Current Assets} make up 32897 of hte total \textit{Current Asset's} value, amounting to 66063.

\end{itemize}
\subsection{Liabilities}
\subsubsection{Non-Current Liabilities}
The analysis of liabilities is threfore reported starting off with non current liabilities, whcih include
\subsubsection{Current Liabilities}
\subsection{Equity}
\section{Income Statement} 
\end{document}
