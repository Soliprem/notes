\documentclass[17pt]{extletter}
\usepackage{extsizes}
\usepackage{LobsterTwo}
\usepackage{graphicx}
\graphicspath{ {../Documents/} }
\usepackage[T1]{fontenc}
\usepackage[margin=1in]{geometry}
% \pagenumbering{gobble}


\begin{document}
\hspace{4in}
\begin{center}
{\huge\bfseries
Una Storia}
\end{center}

Caro Clan,\\
So che non è d'uso scrivere una lettera prima della partenza.\\
Però questo periodo è tutt'altro che ordinario. Ho bisogno di lasciarvi qualcosa che possa essere me nella mia assenza.\\
In realtà vi ho già detto molto nei miei deserti: come lo scoutismo sia il percorso più duraturo che abbia mai intrapreso, di come mi sia difficile allontanarmi da voi, tantopiù adesso che siete in così pochi.\\
C'è una storia, però, che non vi ho mai raccontato, e che forse nessuno sa per intero: vorrei affidarvela.\\
Questa non è una storia che si annacqua col vino attorno al fuoco, nè da raccontare prima della buonanotte: questa storia sono io, e per dir la verità non è neanche una storia per molti orecchi alla volta, e quindi vi darò del Tu.\\
Immagina una fattoria. Un'enorme distesa di piccoli, grandi e vari edifici, ciascuno pieno di anfratti e angoli curiosi. Brulicante di vita. Campi che si spandono a vista d'occhio, cantine piene di otri di vino che invecchia.\\
Il sigillo, se si vuole, di un tempo passato.\\
In questa fattoria immagina una cucciolata di maiali ed uno scherzo del fato, che fa che su di uno dei cuccioli s'incastri un bastone con in fondo una carota magica, che appare sempre più bella mano a mano che invecchia. Ora osserva i maialini che scorrazzano nell'aia in cerca di bocconi, per poi trovare piacere nello sfamarsi, raggiungere il riposo e poi ripartire.\\
Fanno tutto allo stesso modo, meno uno: lui corre dietro la sua carota, a atutto fiato, e non la raggiunge mai. Ogni tanto raccatta dei bocconi, e in realtà finisce per trovarne pure tanti, ma non è mai satollo. Non è mai pronto a riposare.\\
Un po' come una farfalla goffa ed inelegante, scorrazza tentando di afferrare l'impossibile ignoto.\\
Penso sia chiaro che la storia sia autoreferenziale: ho passato gran parte della mia vita ad elucubrare su utopie e grandi sogni irrealizzbili, e per certi versi ancora lo faccio. Però c'è una differenza fra il prima ed il dopo: il tempo.\\
Il tempo, mio malgrado, corre.\\
Gli obbiettivi di un futuro remoto di colpo sembrano vedermi indietro sulla tabella di marcia.\\
Ma c'è un'altra differenza: me stesso.\\
Sono molto cambiato negli anni, ed in buona parte è anche merito tuo.\\
Tuo che mi hai spiegato il valore dei bocconi che trovo scorrazzando in giro.\\
Tuo che mi hai insegnato il valore del riposo.\\
Tuo che, da maialino solitario che ero diventato, mi hai ricordato la bellezza della compagnia amica.\\
Or ti chiedo di immaginare una clessidra con tre posti a sedere, uno in alto, uno in basso, uno nel mezzo.\\
Sei tu ad avermi convinto che vivere in alto, vedere la sabbia scivolare via, porta solo malinconia.\\
Ad avermi convinto che vivere in basso, con la sabbia che ti sotterra, porta solo angoscia.\\
Mi hai mostrato che vedere la sabbia scorrere nel mezzo è l'unica via per la serenità.\\
Però non avevi modo di fare i conti col maiale ideologo: ha zampe forti dal rincorrere la carota per una vita, una capoccia di ghisa e, nei suoi occhi, nulla da perdere.\\
Consapevole di avere la serenità a un passo, continua a correre e correre, e probabilmente così farà per sempre.\\
Voglio poterti ringraziare per avermi mostrato una via alla serenità, anche se non la seguo nella sua interezza, e di non avermela servita su di un piatto d'argento, ma di avermela fatta sudare: è il più bel dono che chiunque potesse farmi.\\
Per certi versi forse è vero che la saggezza non s'insegna, ma s'impara: vista così, sei stato il miglior maestro che io sappia immaginare, proprio perchè senza insegnarmi molto, mi hai fatto imparare tanto.\\
Ora è vero che sarò distante, ma ho un debito da colmare, una carota da inseguire ed una clessidra che scorre inesorabilmente: vorrei che alcuni dei miei granelli vengano spesi al tuo servizio: se c'è una cosa che ho imparato inseguendo una carota per tutta la vita, è che nulla è aldifuori della tua portata, e così tu non sei aldifuori della mia, nè lo sarai mai.\\
Avrai sempre un fratello in me.\\
Ovviamente questo non è che uno scheletro della storia, ma penso dica quello che c'è bisogno di dire: conosci già le mie tristezze, i miei dubbi, le mie paure. Questo non è che un filo conduttore. Spero di rivederti presto.\\
Con Amore, \\\\
\includegraphics[scale=2.0]{sigature}




\end{document}
