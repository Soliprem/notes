\documentclass{article}
\title{Relationship between perception of Political Systems and electoral Law}
\author{Francesco Prem Solidoro}
\usepackage[backend=biber]{biblatex}
\addbibresource{soli_list.bib}
\begin{document}
\maketitle
\section{Abstract} % (fold)
\label{sec:Abstract}
\subsection{Reason why this was made} % (fold)
\label{sub:Reason why this was made}

% subsection subsection name (end)
There's a massive body of literature describing the effect of electoral law on Political systems, but it's not been backed with any substantial research regarding the perception of said systems. The aim of this paper is to study said perception, and what systems best seem to fit the actual desire of the people subject to this paper's survey.

\subsection{Methodology} % (fold)
\label{sec:Methodology}
The survey was sent to over 2000 people, and received 161 answers. The data from these answers has been used to interpret a study form Overseas Development Institute (ODI) ~\cite{ODI-electoral-systems}

% subsection section name (end)

\subsection{Results}
From the data gathered, we reached the conclusion that the most suitable electoral law is a straight proportional system, as it seems that the most commonly preferred number of parties from the respondents was of 5, and according to the ODI study, proportional systems result in the most political parties being relevant. The drawback of proportional electoral systems, it's slowness, also was the lowest ranked by the respondents when asked which problem was the least important in government

% section section name (end)
\printbibliography
	
\end{document}
