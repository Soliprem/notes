\documentclass{article}
\title{Relationship between perception of Political Systems and electoral Law}
\author{Francesco Prem Solidoro}
\usepackage[backend=biber]{biblatex}
\addbibresource{soli_list.bib}
\begin{document}
\maketitle
\section{Abstract} % (fold)
\label{sec:Abstract}
\subsection{Why this was made} % (fold)
\label{sub:Reason why this was made}

% subsection subsection name (end)
There's a massive body of literature describing the effect of electoral law on Political systems, but it's not been backed with any substantial research regarding the perception of said systems. The aim of this paper is to study said perception, and what systems best seem to fit the actual desire of the people subject to this paper's survey.

\subsection{Methodology} % (fold)
\label{sec:Methodology}
The survey 161 answers. The data from these answers has been used to interpret a study made for the Overseas Development Institute (ODI) ~\cite{ODI-electoral-systems}, which establishes a correlation between the amount of parties present in a political system, and its electoral law.

% subsection section name (end)

\subsection{Results}
From the data gathered, we reached the conclusion that the most suitable electoral law is a straight proportional system, as it seems that the most commonly preferred number of parties from the respondents was of 5, and according to the ODI study, proportional systems result in the most political parties being relevant. The drawback of proportional electoral systems, it's slowness, also was the lowest ranked by the respondents when asked which problem was the least important in government

% section section name (end)
\newpage

\section{The Survey}
The survey used had the following questions:\\
\begin{itemize}
  \item what's your gender?\\
  \item What's your highest education level?\\
  \item how old are you?\\
  \item Are you of voting age in your country?\\
  \item What country are you from?\\
  \item where would you put yourself on the political spectrum?\\
  \item "How many (medium-size to large) parties do you think should be present in a healthy political system?\\
  \item (small parties such as the Libertarian party in the United States do not count for the purposes of this survey: the United States in this case would be considered to have 2 parties, even though it technically has more)"\\ \item 
"How many (medium-size to large) parties do you think there are in your country's political system?
\end{itemize}

\subsection{Sample Demographics} % (fold)
\label{sub:Demographic}

% subsection subsection name (end)
\subsubsection{Gender Distribution} % (fold)
\label{subsub:Gender Distribution}
The distribution across genders was heavily slanted towards men (65\%), and slightly less women (29.3\%), with a small component of non-binary people (3\%), a small amount of people who declined to answer (1.8\%).
% subsection subsection name (end)
\subsubsection{Highest Education Level} % (fold)
\label{subsub:Highest Education Level}
Most people reported a highest education level of a High School diploma (37.2\%), followed by Bachelor degree (30.5\%), Masters degree (14.6\%), PhD (9.1\%), and finally lower (8.5\%). Cross referencing with countries and age groups, I was able to determine that most of the components of "lower" are still attending high school by checking the aggregate data of the single respondents that answered "lower" in this question. The aggregate data for each respondent will not be published, as it might be considered profiling.
% subsection subsection name (end)
\subsubsection{Age} % (fold)
\label{subsub:Age}
The age varies, with the most being spread from 14 (1 respondent) to 81 (1 respondent). Most of the answers are within the age group of 18-22.\\
The overwhelming majority (93.9\%) of respondents are of voting age in their country.
% subsection subsection name (end)
\subsubsection{Nationality} % (fold)
\label{subsub:Nationality}
Responses were widely spread across different countries, with different political systems: most of the respondents are from Italy (49.7\%), followed by the USA (11.8\%), Canada (4.3\%), Germany and the UK (3.7\% each), Denmark (3.1\%), Australia (2.4\%), Argentina and the Philippines (1.8\% each). Results came form many other countries, but with less responses.
% subsection subsection name (end)
\subsubsection{Political Orientation} % (fold)
\label{subsub:Political Orientation}
Using the Political Compass, which takes into account economic orientation (Left-Center-Right) and social preference (Authoritarian-Center-libertarian),placed themselves in the Center on the social preference scale, followed by the Libertarian. Across both of these, the distribution remains the same: Most people identified as left-leaning, followed by Center, and finally by right leaning. Of the people who indicated an authoritarian position, most were center, closely followed by left-leaning, and finally by right-leaning.\\
From this we can safely conclude that a left-wing bias is present within this survey's respondents.
% subsection subsection name (end)
\subsubsection{Priorities}
When asked ``What do you value least in government" between Speed(e.g. in passing legislation), Stability,(allowing for long-term goals and plans) and Representation,(properly representing a country's constituents), most responded with Speed (50.9\%), followed by Representation (25.8\%), and finally Stability (23.3\%).

\subsubsection{Field}
Most respondents work or study within a STEM field (42.8\%), followed by Social Sciences (25.8\%), Humanities (13.2\%) and HEAL (health, education, administration, literacy) (7.5\%). The remaining answered others, with an aggregation of 1.8\% for "service", and 3.1\% declined to specify.

\subsection{Parties} % (fold)
\label{sub:Parties}
% subsection subsection name (end)
In this section, when talking about parties we are referring to medium to large parties: therefore in systems like the USA, where smaller parties such as the Libertarian party or the socialist party do exist, only two parties would be considered.
\subsubsection{Preference}
Within the population of the survey, a preference was indicated for 5-party political systems (36\%), followed by 3 (29.3\%), 4 (23.2\%), 2 (9.8\%), and finally 1-party systems (1.2\%)

\subsubsection{Present-day Perception}
When asked how many parties they thought were present in their country, most (46.3\%) stated that there were 5 parties, followed by 20.7\% who reported a 2-party system, most likely coming in large part from the USA respondents, followed by 15.9\% for 3 party-systems, 25\% for 4-party systems, and by 1.8\% for 1-party systems

\subsection{Perception of the Electoral System} % (fold)
\label{sub:Perception of the Electoral System}

% subsection subsection name (end)
\subsubsection{Self-Assesment}
The respondents were asked to grade, from 1 to 5, how well they knew their country's electoral system, and most were quite confident, grading at 4 (32.9\%), followed by 3 at 26.8\%, 5 at 19.5\%, 2 at 15.2\%, and finally 1 at 5.5\%.

\subsubsection{Perception}
When asked ``What's your opinion on your country's electoral system", most answered neutrally or negatively, with 3 coming in at 34.1\%, 2 at 29.3\%, 1 at 17.7\%, 4 at 14.6\%, and 5 at 4.3\%.



\printbibliography
	
\end{document}
